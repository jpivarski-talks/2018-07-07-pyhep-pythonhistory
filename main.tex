\documentclass[aspectratio=169]{beamer}

\mode<presentation>
{
  \usetheme{default}
  \usecolortheme{default}
  \usefonttheme{default}
  \setbeamertemplate{navigation symbols}{}
  \setbeamertemplate{caption}[numbered]
  \setbeamertemplate{footline}[frame number]  % or "page number"
  \setbeamercolor{frametitle}{fg=white}
  \setbeamercolor{footline}{fg=black}
} 

\usepackage[english]{babel}
\usepackage[utf8x]{inputenc}
\usepackage{tikz}
\usepackage{courier}
\usepackage{array}
\usepackage{bold-extra}
\usepackage{minted}
\usepackage[thicklines]{cancel}

\xdefinecolor{dianablue}{rgb}{0.18,0.24,0.31}
\xdefinecolor{darkblue}{rgb}{0.1,0.1,0.7}
\xdefinecolor{darkgreen}{rgb}{0,0.5,0}
\xdefinecolor{darkgrey}{rgb}{0.35,0.35,0.35}
\xdefinecolor{darkorange}{rgb}{0.8,0.5,0}
\xdefinecolor{darkred}{rgb}{0.7,0,0}
\definecolor{darkgreen}{rgb}{0,0.6,0}
\definecolor{mauve}{rgb}{0.58,0,0.82}

\title[2018-07-07-pyhep-pythonhistory]{The Python Scientific Software Ecosystem: \\ Past, Present and Future}
\author{Jim Pivarski}
\institute{Princeton University -- DIANA-HEP}
\date{July 7, 2018}

\begin{document}

\logo{\pgfputat{\pgfxy(0.11, 7.4)}{\pgfbox[right,base]{\tikz{\filldraw[fill=dianablue, draw=none] (0 cm, 0 cm) rectangle (50 cm, 1 cm);}\mbox{\hspace{-8 cm}\includegraphics[height=1 cm]{princeton-logo-long.png}\includegraphics[height=1 cm]{diana-hep-logo-long.png}}}}}

\begin{frame}
  \titlepage
\end{frame}

\logo{\pgfputat{\pgfxy(0.11, 7.4)}{\pgfbox[right,base]{\tikz{\filldraw[fill=dianablue, draw=none] (0 cm, 0 cm) rectangle (50 cm, 1 cm);}\mbox{\hspace{-8 cm}\includegraphics[height=1 cm]{princeton-logo.png}\includegraphics[height=1 cm]{diana-hep-logo.png}}}}}

% Uncomment these lines for an automatically generated outline.
%\begin{frame}{Outline}
%  \tableofcontents
%\end{frame}

% START START START START START START START START START START START START START

\begin{frame}{Why we're here}
\vspace{0.25 cm}
\begin{center}
\includegraphics[width=0.8\linewidth]{pypl-popularity.png}
\end{center}
\textcolor{blue}{\scriptsize\url{http://pypl.github.io/PYPL.html}}
\end{frame}

\begin{frame}{Why we're here}
\vspace{0.5 cm}
\includegraphics[width=\linewidth]{python-r-cpp-googletrends-data.png}

\vspace{1 cm}
\includegraphics[width=\linewidth]{python-r-cpp-googletrends-dataset.png}
\end{frame}

\begin{frame}{Why we're here}
\vspace{0.5 cm}
\includegraphics[width=\linewidth]{python-r-cpp-googletrends-datascience.png}

\vspace{1 cm}
\includegraphics[width=\linewidth]{python-r-cpp-googletrends-machinelearning.png}
\end{frame}

\begin{frame}{Why we're here}
\large
\vspace{0.4 cm}
All of the machine learning libraries I could find either have a Python interface or are primarily/exclusively Python.

\vspace{0.6 cm}
\mbox{ } \includegraphics[height=0.8 cm]{sklearn-logo.png}
\hfill \includegraphics[height=0.8 cm]{pytorch-logo.png}
\hfill \includegraphics[height=0.8 cm]{keras-logo.png}
\hfill \includegraphics[height=1 cm]{tensorflow-logo.png}
\hfill \includegraphics[height=0.8 cm]{caffe2-logo.png}
\hfill \includegraphics[height=0.8 cm]{gluon-logo.png} \mbox{ }

\vspace{0.15 cm}
\mbox{ } \includegraphics[height=0.8 cm]{chainer-logo.png}
\hfill \includegraphics[height=0.8 cm]{cntk-logo.png}
\hfill \includegraphics[height=0.8 cm]{lasagne-logo.png}
\hfill \includegraphics[height=0.8 cm]{onnx-logo.png}
\hfill \includegraphics[height=0.8 cm]{cesium-logo.png}
\hfill \includegraphics[height=0.8 cm]{xgboost-logo.png} \mbox{ }
\end{frame}

\begin{frame}{Why we're here}
\vspace{0.25 cm}
\begin{center}
\includegraphics[width=0.7\linewidth]{mentions-of-programming-languages.png}
\end{center}
\end{frame}

\begin{frame}{Why we're here}
\large
\vspace{0.5 cm}
\textcolor{darkblue}{\huge Python in HEP?}

\vspace{0.5 cm}
\begin{itemize}\setlength{\itemsep}{0.5 cm}
\item<2-> Collaboration frameworks like Athena and CMSSW are configured or driven by Python.

\item<3-> Today, it is common for physicists to do their analyses in Python/PyROOT.

\vspace{0.1 cm}
\textcolor{gray}{\normalsize (Half Python, half C++? Anyway, a lot more than in LHC Run I.)}

\item<4-> Python is a bridge to machine learning and other statistical software written outside of HEP.
\end{itemize}
\end{frame}

\begin{frame}{}
\Large
\vspace{1 cm}
\textcolor{darkgray}{To be honest, Python isn't my favorite language.}

\large
\vspace{1 cm}
\uncover<2->{\textcolor{darkgray}{I wish the scientific and data analysis community had adopted something more functional and statically typed. In particular, I wish there wasn't a dichotomy between statements and expressions. And {\tt True == 1} is evil.}}

\large
\vspace{1 cm}
\uncover<3->{\textcolor{darkgray}{But a dash of consensus is well worth a smattering of language features\ldots}}
\end{frame}

\begin{frame}{Why Python?}
\large
\vspace{0.5 cm}
\begin{columns}[t]
\column{0.55\linewidth}
\underline{Did Python just arrive at the right time?}

\vspace{0.25 cm}
\begin{uncoverenv}<2->
\begin{itemize}
\item Ruby, Lua numerical stacks were not ready before Python already had a foothold.
\item Python was one of the first glue languages of the Linux/open source era.
\end{itemize}
\end{uncoverenv}

\column{0.5\linewidth}
\underline{Or is it a better language for the job?}

\vspace{0.25 cm}
\begin{uncoverenv}<3->
\begin{itemize}
\item Perl $\to$ Python
\item ``Tcl War''
\item R $\to$ Python
\item<4-> {\it An Empirical Investigation into Programming Language Syntax,} Andreas Stefik, Susanna Siebert.
\end{itemize}
\end{uncoverenv}
\end{columns}
\end{frame}

\begin{frame}{An Empirical Investigation into Programming Language Syntax}
\vspace{0.25 cm}
\begin{columns}
\column{0.58\linewidth}
\includegraphics[width=\linewidth]{empirical-examples.png}

\column{0.42\linewidth}
\includegraphics[width=\linewidth]{empirical-abstract.png}

\vspace{1 cm}
\mbox{\hspace{-1 cm}\includegraphics[width=1.25\linewidth]{empirical-table.png}}
\end{columns}
\end{frame}

\begin{frame}{My conclusion (debatable)}
\huge
\vspace{1 cm}
\begin{center}
Python was good enough and first.
\end{center}
\end{frame}

\begin{frame}{}


\end{frame}



\end{document}


%% https://speakerdeck.com/jakevdp/the-unexpected-effectiveness-of-python-in-science

